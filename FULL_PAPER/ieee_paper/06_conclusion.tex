
The \texttt{move\_car} system successfully demonstrates robust autonomous navigation capabilities through the 
tight integration of modern perception algorithms with classical planning and control techniques. 
Our comprehensive evaluation confirms its real-time performance, making it suitable for practical deployment in 
dynamic environments. Furthermore, its modular architecture ensures adaptability and facilitates future extensions 
and improvements.

Key advantages of the \texttt{move\_car} system include its direct multi-modal sensor fusion approach, 
which reduces reliance on costly high-definition maps, and its proven robust operation in complex, 
real-world driving scenarios within a simulated environment.

Specifically, the \texttt{move\_car} ADAS stack achieves a balanced trade-off between latency, accuracy, 
and computational complexity, as evidenced in our simulated tests:

\begin{itemize}
    \item \textbf{Real-Time Performance:} PointPillars, integrated within our perception module, excels with its 
    fast inference speed (6.84 ms) and balanced accuracy (77.00 Car@R11), proving highly effective for real-time 
    applications.
    \item \textbf{High-Precision Tasks:} For scenarios demanding higher precision, CenterPoint focuses on accurate 
    object localization and classification, demonstrating its capability for detailed environmental understanding.
    \item \textbf{Complex Environments:} BEVFusion significantly leverages multi-modal fusion of LiDAR and camera 
    data, providing superior and robust perception, particularly in challenging and complex driving conditions.
\end{itemize}

Building upon these foundational achievements, future work for the \texttt{move\_car} 
project will explore several key areas to further enhance its capabilities. 
These include the integration of advanced anomaly detection mechanisms for improved safety in critical edge cases,
 the incorporation of language-driven planning modules to enable more intuitive and context-aware high-level 
 decision-making, and the enhancement of semantic alignment with road topology, potentially utilizing HD maps 
 or advanced visual cues for increased navigational precision.
